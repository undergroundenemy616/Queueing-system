\documentclass[a4paper,11pt]{report}
\usepackage{multicol}
\usepackage{color}
\usepackage{listings}
\usepackage{amsmath,amsthm,amssymb}
\usepackage{graphicx}
\usepackage[left=2.5cm, right=1.5cm, vmargin=2.5cm]{geometry}
\usepackage[T1,T2A]{fontenc}
\usepackage[utf8]{inputenc}
\usepackage[english,russian]{babel}

\begin{document}
\setcounter{secnumdepth}{-1}

\section{Исходные данные к работе}
\subsection{1. Характеристика входного потока}
\begin{itemize}
  \item \underline{Тип источников} \textcolor{blue}{БЕСКОНЕЧНЫЙ}
  \newline источник генерирует заявку, а затем определяет интервал (по детерминированному или случайному закону) для генерации следующей заявки. Таким образом, заявка попадает в систему в момент генерации и проходит по ней свой индивидуальный путь
  \item \underline{Количество источников} \textcolor{red}{INPUT}
  \item \underline{Закон генерации заявок} \textcolor{blue}{РАВНОМЕРНЫЙ(закон распределения) - через равные промежутки времени? или это INPUT на детерминированный / случайный}
  \item \underline{Среднее время между заявками} \textcolor{red}{INPUT ???}
  \item \underline{Приоритеты входных потоков} \textcolor{blue}{см. 2. Характеристики ВС}
\end{itemize}

\subsection{2. Характеристики ВС}
\begin{itemize}
  \item Буферная память
  \begin{itemize}
    \item Тип буфера \textcolor{blue}{см. Дисциплины диспетчеризации}
    \item Объём буфера \textcolor{red}{INPUT}
  \end{itemize}
  \item Обслуживающие приборы
  \begin{itemize}
    \item Количество приборов \textcolor{red}{INPUT}
    \item Закон обслуживания \textcolor{blue}{ЭКСПОНЕНЦИАЛЬНЫЙ - то есть первую за e, вторую за $e^2$ ... ?}
    \item Среднее время обслуживания \textcolor{red}{INPUT ???}
  \end{itemize}
  \item Дисциплины диспетчеризации (Диспетчер постановки и Диспетчер выбора)
  \begin{itemize}
    \item \underline{Дисциплины буферизации} \textcolor{blue}{В ПОРЯДКЕ ПОСТУПЛЕНИЯ}
    \newline Если в момент поступления заявок в систему все приборы оказываются занятыми, заявка последовательно занимает места в буфере памяти, начиная с первого. В случае освобождения какого-либо места в БП с номером N (заявка уходит на обслуживание или получает отказ), все заявки, стоящие на местах, начиная с (N+1), сдвигаются на одно место. Следующая заявка, вынужденная встать в очередь, всегда будет ставиться в ее конец, пока есть свободные места.
    \item \underline{Дисциплина отказа (выбивания)} \textcolor{blue}{ПРИОРИТЕТ ПО НОМЕРУ ИСТОЧНИКА}
    \newline При этой дисциплине отказ получает заявка с наименьшим приоритетом среди тех, что на данный момент находятся в БП (приоритет заявки определяется номером источника, который её сгенерировал). Если к этому времени в буфере имеется несколько заявок от источника с минимальным приоритетом, то встаёт вопрос, какую из этих заявок отправить в отказ.
    \newline При этом необходимо вспомнить, что заявка имеет две характеристики: номер источника и время генерации. В нашем случае первая характеристика у этих заявок одинаковая и различаются они только временем генерации. Если в рассматриваемой ВС выбор заявки на обслуживание происходит по времени (FIFO, LIFO), то это может оказаться подсказкой для определения очерёдности при отказе.
    \item \underline{Дисциплина выбора заявок на обслуживание} \textcolor{blue}{ПРИОРИТЕТ ПО НОМЕРУ ИСТОЧНИКА, ЗАЯВКИ В ПАКЕТЕ}
    \newline Назовем  «пакетом»  совокупность  заявок  одного  источника, находящихся в буфере на момент освобождения одного из приборов.
    \newline Количество  пакетов  в  БП  может  меняться  от  0  до  n,  где  n  — количество источников.
    \newline Когда при освобождении прибора происходит выбор первой заявки из буфера, вначале определяется самый приоритетный на данный момент пакет и происходит обслуживание заявок только этого пакета до тех пор, пока к моменту очередного освобождения прибора в БП не останется ни одной заявки этого пакета. Затем снова определяется самый приоритетный на данный момент пакет и далее повторяется весь процесс обслуживания этого пакета. Таким образом, происходит динамическая смена приоритетов обслуживания заявок, причем приоритетность пакетов можно регулировать, изменяя интенсивность генерации заявок источниками.
    \item \underline{Дисциплина занятия устройств} \textcolor{blue}{ПО КОЛЬЦУ}
    \newline Эта дисциплина производит выбор свободного прибора таким же способом, как и аналогичная дисциплины выбора заявок из буфера по кольцу, т. е. поиск свободных приборов каждый раз начинается с указателя и заявка встает на обслуживание на первый из найденных приборов.
    \newline СТР. 40 ПОСОБИЯ 1.
  \end{itemize}
\end{itemize}

\subsection{3. Характер текущей отображаемой информации (пошаговый режим)}
\begin{itemize}
  \item Календарь событий, буфер, текущее состояние
  \item Формализованная схема модели, текущее состояние
  \item Временные диаграммы, текущее состояние \textcolor{blue}{МОЙ ВАРИАНТ}
\end{itemize}

\subsection{4. Представление окончательных результатов моделирования (автоматический режим)}
\begin{itemize}
  \item Сводная таблица результатов \textcolor{blue}{МОЙ ВАРИАНТ}
  \item Основные графики (графики зависимости основных входных характеристик моделируемой системы от
                          изменения перечисленных выше входных параметров)
\end{itemize}


\section{Требования к результатам}
В работе требуется построить моделирующий алгоритм ВС, реализовать его на языке программирования, отладить модель и
представить в отчете результаты, полученные с относительной точностью 10\% и доверительной вероятностью 0.9,
сопровождаемые анализом, графиками и выводами. \newline
Таким образом, необходимо получить и проанализировать следующие характеристики ВС в зависимости от общей загрузки системы (p):
\noindent
\begin{enumerate}
  \item количество сгенерированных каждым источником требований
  \item вероятность отказа в обслуживании заявок (требований) каждого источника ( $P_{otk}$ )
  \item среднее время пребывания заявок каждого источника в системе ( $T_{sys}$)
  \item среднее время ожидания заявок каждого источника в системе ( $T_{wait}$ )
  \item среднее время обслуживания заявок каждого источника ( $T_{service}$ )
  \item дисперсии двух последних характеристик
  \item коэффициент  использования  устройств  (  $K_{use}$ )  (время работы каждого прибора / время реализации).
\end{enumerate}

\noindent
Эта часть задания относится к исследованию формально заданной ВС, у которой значения входным характеристикам задаются произвольно и определяются требованиями к выходным характеристикам ВС.\newline
Следующим этапом работы является процесс исследования конкретной вычислительной системы (или ее компоненты) на полученной ранее модели с заданным уровнем доверительной вероятности и точности результатов.\newline
На основе этого исследования, ориентируясь на предъявляемые требования к результатам работы системы, т. е. к ее выходным характеристикам, необходимо найти нужную архитектуру системы, состоящую из ее элементов и работающую в соответствии с заданными требованиями.\newline
Элементами ВС в нашем случае являются входные параметры.



\end{document}
